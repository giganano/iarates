
\documentclass[ms.tex]{subfiles}
\begin{document}

\begin{abstract}
The scaling of the specific type Ia supernova (SN Ia) rate with host galaxy
stellar mass~$\dot{\text{N}}_\text{Ia} / \mstar \sim \mstar^{-0.3}$ strongly
suggests that the number of events produced by a stellar population depends on
its metallicity.
We estimate the strength of the required metallicity dependence by combining
the average star formation histories (SFHs) of galaxies as a function of their
stellar mass with the mass-metallicity relation (MZR) for galaxies and common
parametrizations for the SN Ia delay-time distribution.
The differences in SFHs can account for only~$\sim$30\% of the increase in the
specific SN Ia rate between~$10^{7.2}$ and~$10^{10}~\msun$.
We find that a metallicity dependence of approximately~$\sim$Z$^{-0.5}$ is
required to explain the observed scaling.
This scaling matches the metallicity dependence of the close binary fraction
observed in APOGEE, suggesting that the enhanced SN Ia rate in low-mass
galaxies can be explained by a combination of their more extended SFHs and a
higher binary fraction due to their lower metallicities.
Due to the shape of the MZR, only galaxies below
$\mstar\approx3\times10^9~\msun$ are significantly affected by
metallicity-dependent SN Ia rates.
The~$\dot{\text{N}}_\text{Ia} / \mstar \sim \mstar^{-0.3}$ scaling becomes
shallower with increasing redshift, dropping by a factor of~$\sim$3 at
$10^{7.2}~\msun$ with metallicity-independent SN Ia rates but only by a factor
of~$\sim$2 if $\dot{\text{N}}_\text{Ia} \sim Z^{-0.5}$.
We briefly discuss the implications of metallicity-dependent SN Ia rates in
one-zone models of galactic chemical evolution.
\end{abstract}

\end{document}
