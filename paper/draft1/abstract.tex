
\documentclass[ms.tex]{subfiles}
\begin{document}

\begin{abstract}
Based on the scaling of the specific type Ia supernova (SN Ia) rate with host
galaxy stellar mass, recent studies have suggested that the number of events
produced by a stellar population may depend on its metallicity.
We estimate the strength of the metallicity-dependence required to explain the
observed SN rates by combining the mass-metallicity relation and simple
assumptions regarding the delay-time distribution of SNe Ia with the average
star formation history (SFH) as a function of stellar mass predicted by
the~\um~semi-analytic model of galaxy formation.
In agreement with previous results, we find that the differences in SFHs can
account for only a factor of~$\sim$2 increase in the specific SN Ia rate
between~$\mstar \approx 10^{10}~\msun$ and~$\mstar \approx 10^{7.2}~\msun$.
For galaxy stellar mass functions (SMFs) that are steep at the low-mass end, an
additional metallicity-dependence of~$Z^{-0.5}$ is required to explain the
observed rates, while shallow SMFs require a considerably stronger scaling of
$\sim Z^{-1.5}$ due to the lower abundance of the underlying galaxy population.
We show that~$Z^{-0.5}$ agrees well with the metallicity-dependence of the
close binary fraction observed in APOGEE, suggesting that the enhanced SN Ia
rate in low-mass galaxies can be explained by having more progenitors due to
having a lower metallicity and consequently a higher binary fraction.
We discuss the prospects of observational diagnostics of metallicity-dependent
SN Ia rates or lack thereof, including the scaling of the specific SN Ia rate
as a function of stellar mass at redshift~$z = 1$ and the stellar mass
distribution of SN Ia host galaxies.
\end{abstract}

\end{document}
