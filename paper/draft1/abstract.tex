
\documentclass[ms.tex]{subfiles}
\begin{document}

\begin{abstract}
Based on the scaling of the specific type Ia supernova (SN Ia) rate with host
galaxy stellar mass, recent studies have suggested that the number of events
produced by a stellar population may depend on its metallicity.
We estimate the strength of the metallicity dependence required to explain the
observed SN Ia rates between~$10^{7.2}$ and~$10^{12}~\msun$ by combining the
average star formation histories (SFHs) of galaxies as a function of their
stellar mass with the mass-metallicity relation for galaxies and common
parametrizations for the SN Ia delay-time distribution.
The differences in SFHs can account for only a factor
of~$\sim$2 increase in the specific SN Ia rate between~$10^{7.2}$
and~$10^{10}~\msun$.
Due to the shape of the mass-metallicity relation, only galaxies below
$\mstar\approx3\times10^9~\msun$ are significantly affected by
metallicity-dependent SN Ia rates.
In agreement with previous results, we find that a metallicity dependence of
approximately~$\sim Z^{-0.5}$ is required to explain an observed scaling of
$\dot{\text{N}}_\text{Ia} / \mstar \sim \mstar^{-0.3}$.
Steeper scalings such as~$\dot{\text{N}}_\text{Ia} / \mstar \sim \mstar^{-0.5}$
require even stronger scalings with metallicity of~$\sim Z^{-1.5}$.
We show that a~$Z^{-0.5}$ scaling agrees well with the metallicity dependence
of the close binary fraction observed in APOGEE, suggesting that the enhanced
SN Ia rate in low-mass galaxies can be explained by a combination of their more
extended SFHs and a higher binary fraction due to their
lower metallicities.
If the~$\dot{\text{N}}_\text{Ia} / \mstar \sim \mstar^{-0.3}$ scaling is
accurate, then SFHs and binary fractions alone can explain the observed results
while stronger scalings require additional effects.
\end{abstract}

\end{document}
