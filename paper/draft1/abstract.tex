
\documentclass[ms.tex]{subfiles}
\begin{document}

\begin{abstract}
The scaling of the specific Type Ia supernova (SN Ia) rate with host galaxy
stellar mass~$\dot{\text{N}}_\text{Ia} / \mstar \sim \mstar^{-0.3}$ as measured
in ASAS-SN and DES strongly suggests that the number of SNe Ia produced by a
stellar population depends inversely on its metallicity.
We estimate the strength of the required metallicity dependence by combining
the average star formation histories (SFHs) of galaxies as a function of their
stellar mass with the mass-metallicity relation (MZR) for galaxies and common
parametrizations for the SN Ia delay-time distribution.
The differences in SFHs can account for only~$\sim$30\% of the increase in the
specific SN Ia rate between stellar masses of~$10^{10}$ and~$10^{7.2}~\msun$.
We find that an additional metallicity dependence of
approximately~$\sim$Z$^{-0.5}$ is required to explain the observed scaling.
This scaling matches the metallicity dependence of the close binary fraction
observed in APOGEE, suggesting that the enhanced SN Ia rate in low-mass
galaxies can be explained by a combination of their more extended SFHs and a
higher binary fraction due to their lower metallicities.
Due to the shape of the MZR, only galaxies below
$\mstar\approx3\times10^9~\msun$ are significantly affected by the
metallicity-dependent SN Ia rates.
The~$\dot{\text{N}}_\text{Ia} / \mstar \sim \mstar^{-0.3}$ scaling becomes
shallower with increasing redshift, dropping by factor of~$\sim$2
at~$10^{7.2}~\msun$ between~$z = 0$ and~$1$ with our~$\sim$$Z^{-0.5}$ scaling.
With metallicity-independent rates, this decrease is a factor of~$\sim$3.
We discuss the implications of metallicity-dependent SN Ia rates for one-zone
models of galactic chemical evolution.
\end{abstract}

\end{document}
