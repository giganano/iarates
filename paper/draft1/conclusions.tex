
\documentclass[ms.tex]{subfiles}
\begin{document}

\section{Discussion and Conclusions}
\label{sec:conclusions}

We have combined the mean SFHs of galaxies as predicted by the~\um~semi-analytic
model of galaxy formation~\citep{Behroozi2019} with the empirical MZR as
parametrized by~\citet{Zahid2014}.
Under the assumption that metallicity effects are the reason for the high
specific SN Ia rates observed in ASAS-SN~\citep{Brown2019, Gandhi2022} and
DES~\citep{Wiseman2021}, we have conducted simple numerical calculations to
investigate the origin of this effect.
Empirical constraints on the mass dependence of the specific SN Ia rate are
subject to uncertainties in the galaxy SMF at low stellar masses
(\citealp{Gandhi2022}; see also discussion of challenges in SMF measurements
in, e.g.,~\citealp{Weigel2016}).
Within our framework, however, we can theoretically predict the scaling of the
specific SN Ia rate with galaxy stellar mass in a manner that is independent of
the SMF (see equation~\ref{eq:specia}).
We find that the mean SFHs of galaxies of different stellar masses can account
for only a factor of~$\sim$2 increase in the specific SN Ia rate between
$10^{10}$ and~$10^{7.2}~\msun$.
If the rate scales with stellar mass as~$\dot{N}_\text{Ia} / M_\star \sim
M_\star^{-0.3}$, as suggested by Gandhi et al.'s~\citeyearpar{Gandhi2022}
re-scaling of Brown et al.'s~\citeyearpar{Brown2019} measurements from the
\citet{Bell2003} SMF to the~\citet{Baldry2012} SMF, then a metallicity
dependence of~$Z^\gamma$ with~$\gamma \approx -0.5$ is required.
If instead the steeper scaling of~$\dot{N}_\text{Ia} / M_\star \sim
M_\star^{-0.5}$ derived by~\citet{Brown2019} is accurate, then stronger
scalings ($\gamma \approx -1.5$) are required, but~\citet{Gandhi2022}
demonstrate that these scalings fail to reproduce the well-known correlations
of [Mg/Fe] and [Fe/H] abundances.
Based on similar arguments, the realization that SN Ia rates could depend on
metallicity with a~$\gamma = -0.5$ dependence should have important
implications for chemical evolution models of dwarf galaxies, known empirically
to host both stellar populations~\citep{Gallazzi2005, Kirby2013} and gas
reservoirs~\citep{Tremonti2004, Zahid2011, Andrews2013, Zahid2014} with
significantly sub-solar abundances.
SNe Ia are responsible for a substantial portion ($\sim$half) of the abundances
of iron-peak elements in the universe~\citep[e.g.][]{Johnson2019}, and
incorporating more enrichment events at low metallicities could significantly
impact inferred evolutionary parameters in statistical fits to observed data.
\par
A scaling of~$\gamma = -0.5$ is in excellent agreement with the dependence of
the close binary fraction measured in APOGEE~\citep{Moe2019}, which suggests
that if a scaling of~$\dot{N}_\text{Ia} / M_\star \sim M_\star^{-0.3}$ is
accurate, then the elevated SN Ia rates in dwarf galaxies can be explained by a
combination of their more extended SFHs and an increased binary fraction
compared to their higher mass counterparts due to differences in their metal
content.
While~\citet{Gandhi2022} motivate their investigation from this viewpoint, here
we take this argument one step further and postulate that this accounts for
the~\textit{entire} increase in the specific SN Ia rate.
This interpretation goes against the hypothesis raised by~\citet{Kistler2013}
based on massive WDs at low metallicities, suggesting that this effect is
subdominant compared to the increase in the binary fraction, which naturally
accounts for a factor of~$\sim$3 over the~$\sim$1.5 decades in metallicity
spanned by~$10^{7.2} - 10^{10}~\msun$ galaxies.
Nonetheless, massive WDs at low metallicity may be an important component in
the theoretical understanding of the specific SN Ia rate as a function of
stellar mass if the steeper~$\sim M_\star^{-0.5}$ scaling from~\citet{Brown2019}
is accurate.
\par
The calculations we have presented here are simplified in several regards.
First, we have assumed the characteristic SFH predicted by a semi-analytic
model of galaxy formation at all stellar masses.
This approximation is reasonable for computing expected means trends, but in
principle, the underlying galaxy population should occupy a distribution of
SFHs.
\citet{Brown2019} do not find strong evidence that the specific SN Ia rate
depends on star formation activity at the time of observation, but we do not
investigate this here because tracking galaxy-by-galaxy SFHs would pose an
increase in our computational expense of multiple orders of magnitude.
Second, we have assumed that all galaxies lie perfectly on the MZR, which
should also be sufficient for computing mean trends.
However, similar to their SFHs and largely as a consequence thereof, galaxies
should populate a distribution of metallicities at fixed stellar mass.
Third, our metallicity scaling of~$Z^\gamma$ where~$Z$ is taken from the
\citet{Zahid2014} MZR assumes that all SNe Ia in any galaxy arise from stellar
populations whose metallicities are exactly that of the interstellar
medium implied by the MZR.
This approximation is also reasonable because models of galactic chemical
evolution suggest that galaxies should rapidly approach an equilibrium
abundance in which the production of new metals is balanced by dilution due to
accretion of metal-poor gas and losses to star formation and outflows
\citep{Larson1972, Weinberg2017}.
Characterizing metal abundances as an equilibrium phenomenon would suggest that
the young stellar populations which dominate the SN Ia rate (see
Fig.~\ref{fig:sfh_mzr} and discussion in~\S~\ref{sec:galprops}) should be of
similar metallicities -- at least on average.
Although the close binary fraction is known to increase at low abundances
in APOGEE~\citep{Badenes2018, Moe2019},~\citet{Mazzola2020} find that the
binary fraction decreases for stars with high [$\alpha$/Fe] ratios at fixed
[Fe/H], suggesting that the detailed chemical composition may significantly
impact stellar multiplicity.
Furthermore, the abundances of metals in stars is a many-dimensional quantity,
even when grouping them based on their various nucleosynthetic sources
\citep{Ting2022}.
However, a theoretical understanding of the metallicity dependence of SN Ia
rates is a relatively new topic in the literature, and a first-order
understanding of how the rates scale with the absolute metal abundance in
statistically average galaxies must be attained before considering how it
should scale with the detailed chemical composition of the progenitor
populations.

\end{document}
