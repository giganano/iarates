
\documentclass[ms.tex]{subfiles}
\begin{document}

\section{Introduction}
\label{sec:intro}

Type Ia supernovae (SNe Ia) arise from the thermonuclear detonation of a white
dwarf~\citep[WD;][]{Hoyle1960, Colgate1969}, the exposed carbon-oxygen core of
a low-mass star.
SN surveys have revealed that low-mass galaxies are more efficient producers
of these events than their higher mass counterparts~\citep[e.g.,][]{Mannucci2005,
Sullivan2006, Li2011, Smith2012}.
In particular,~\citet{Brown2019} found that the specific SN Ia rate -- the rate
per unit stellar mass -- scales approximately with the inverse square root of
the stellar mass itself ($\dot{\text{N}}_\text{Ia} \mstar \sim \mstar^{0.5}$)
using SNe Ia from the All-Sky Automated Survey for Supernovae
\citep[ASAS-SN;][]{Shappee2014, Kochanek2017} and assuming the~\citet{Bell2003}
stellar mass function (SMF).
% using SNe Ia observed by the All-Sky Automated Survey for
% Supernovae~\citep[ASAS-SN;][]{Shappee2014, Kochanek2017},~\citet{Brown2019}
% found that the specific SN Ia rate -- the rate per unit stellar mass -- scales
% approximately with the inverse square root of the stellar mass itself
% ($\dot{\text{N}}_\text{Ia} / \mstar \sim \mstar^{-0.5}$) assuming the
% \citet{Bell2003} SMF.
The scaling becomes shallower ($\dot{\text{N}}_\text{Ia} / \mstar \sim
\mstar^{-0.3}$) when using the steeper~\citet{Baldry2012} double-Schechter
parametrization due to a more abundant underlying galaxy population at low
stellar masses~\citep{Gandhi2022}.
This change leads to significantly better agreement between the ASAS-SN
measurements and Wiseman et al.'s~\citeyearpar{Wiseman2021} rates measured with
SNe Ia from the Dark Energy Survey~\citep[DES;][]{DES2016}, also normalized
with the~\citet{Baldry2012} SMF.
Although there is debate surrounding the strength of the scaling, it is now
well-established qualitatively that the specific SN Ia rate is higher in dwarf
galaxies.
There are a handful of potential pathways which could give rise to this
empirical result.
\par
First, the mean star formation histories (SFHs) of galaxies vary with the
stellar mass of the system.
In semi-analytic models of galaxy formation (see, e.g., the reviews of
\citealt{Baugh2006} and~\citealt{Somerville2015a}), dwarf galaxies in the field
have more extended SFHs than their higher mass counterparts.
{\color{red} This is also seen in hydrodynamical simulations of galaxy
formation (refs).}
Since SN Ia delay-time distributions (DTDs) decline with age, galaxies with
more recent star formation should have higher specific SN Ia rates.
\par
Second,~\citet{Kistler2013} argued that the dependence of the specific SN Ia
rate on stellar mass may be driven by metallicity.
Lower mass galaxies host lower metallicity stellar populations
\citep{Gallazzi2005, Kirby2013} and lower metallicity gas reservoirs
(\citealp{Tremonti2004};~\citealp*{Zahid2011};~\citealp{Andrews2013,
Zahid2014}).
\citet{Kistler2013} point out that low metallicity stars leave behind higher
mass WDs which could potentially grow to the Chandrasekhar mass and
subsequently explode more easily than their less massive counterparts.
Lower metallicity stars have weaker winds during the asymptotic giant branch
phase~\citep{Willson2000, Marigo2007}, leading to lower mass loss rates and
more massive cores~\citep{Kalirai2014}, producing more massive WDs for fixed
initial mass stars at lower metallcity (\citealp{Umeda1999};
\citealp*{Meng2008};~\citealp{Zhao2012}).
\par
Furthermore, in both the single~\citep[e.g.,][]{Whelan1973} and double
degenerate scenarios (e.g.,~\citealp{Iben1984};~\mbox{\citealp{Webbink1984}}),
SN Ia arise in binary systems.
Based on multiplicity measurements of solar-type stars in APOGEE\footnote{
	Apache Point Observatory Galaxy Evolution Experiment
}~\citep{Majewski2017},~\citet*{Moe2019} found that the stellar close binary
fraction increases from~$\sim$10\% at~$\sim$3 times the metallicity of the sun
$Z_\odot$ to~$\sim$40\% at~$\sim$$0.1 Z_\odot$.
Consequently, dwarf galaxies should have more potential SN Ia progenitors per
unit mass of star formation due to more massive WDs and a higher close binary
fraction.
Motivated by these results,~\citet{Gandhi2022} explore a handful of
parametrizations for the metallicity dependence of the SN Ia rate in
re-simulated galaxies from FIRE-2~\citep{Hopkins2018}.
They find that a~$Z^{-0.5}$ scaling leads to better agreement with the empirical
relationship between galactic stellar masses and stellar abundances than when
using metallicity-independent SN Ia rates.
\par
% At first glance, an inverse dependence of SN Ia rates on metallicity may seem
% at odds with the recent realization by~\citet{Holoien2022} that dwarf galaxy
% hosts of SNe Ia detected by ASAS-SN tend to be oxygen-rich relative to similar
% mass galaxies.
% However, SN hosts detected by any survey are likely not a representative sample
% of the underlying galaxy population because, due to the steeply declining
% nature of the DTD~\citep[e.g.][]{Maoz2012a}, the intrinsically highest SN Ia
% rates should be in systems which experienced a recent starburst ($\lesssim$1
% Gyr ago).
% Since oxygen is produced mostly by massive stars with lifetimes shorter than
% the delay-times of SNe Ia~\citep*[e.g.][]{Hurley2000, Johnson2019}, these
% galaxies should also have a higher-than-average oxygen abundance (for
% theoretical discussion of starbursts in galactic chemical evolution models,
% see, e.g.,~\citealt{Johnson2020}).
% In other words, observed SN Ia hosts may be slightly metal-rich on average, but
% one may arrive at a different conclusion if it were feasible to determine rates
% within individual galaxies.
In this paper, we assume that the strong scaling of the specific SN Ia rate
with stellar mass is due to metallicity effects and conduct simple numerical
calculations to investigate its origin.
We combine the mean star formation histories of galaxies at fixed stellar
mass from the~\um~semi-analytic model~\citep{Behroozi2019} and the popular
$\tau^{-1}$ SN Ia DTD~\citep[e.g.,][]{Maoz2012a} with the mass-metallicity
relation (MZR) for galaxies~\citep{Tremonti2004, Andrews2013, Zahid2011,
Zahid2014}.
With the mean SFH and DTD, we can compute the characteristic SN Ia rate for
galaxies of a given stellar mass, and by assuming that they lie along the
observed MZR, we can simultaneously fold in various scalings with
metallicity~$Z$.

\end{document}
