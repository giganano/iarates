
\documentclass[ms.tex]{subfiles}
\begin{document}

\section{Introduction}
\label{sec:intro}

Type Ia supernovae (SNe Ia) arise from the thermonuclear detonation of a white
dwarf~\citep[WD;][]{Hoyle1960, Colgate1969}, the exposed carbon-oxygen core of
a low-mass star.
SN surveys have revealed that low-mass galaxies are more efficient producers
of these events than their higher mass counterparts~\citep[e.g.,][]{Mannucci2005,
Sullivan2006, Li2011, Smith2012}.
In particular,~\citet{Brown2019} found that the specific SN Ia rate -- the rate
per unit stellar mass -- scales approximately with the inverse square root of
the stellar mass itself ($\dot{\text{N}}_\text{Ia} / \mstar \sim \mstar^{-0.5}$)
using SNe Ia from the All-Sky Automated Survey for Supernovae
\citep[ASAS-SN;][]{Shappee2014, Kochanek2017} and assuming the~\citet{Bell2003}
stellar mass function (SMF).
The measurement depends on the SMF because the number of
observed SNe must be normalized by the number of galaxies in order to compute a
specific SN rate.
Consequently, the scaling becomes shallower ($\dot{\text{N}}_\text{Ia} / \mstar
\sim \mstar^{-0.3}$) when using the steeper~\citet{Baldry2012} double-Schechter
SMF parametrization~\citep{Gandhi2022}.
This change leads to agreement between the ASAS-SN measurements and Wiseman et
al.'s~\citeyearpar{Wiseman2021} estimates from the Dark Energy
Survey~\citep[DES;][]{DES2016} using the~\citet{Baldry2012} SMF.
Although the exact strength of the scaling depends on the SMF, it is clear
that the specific SN Ia rate is higher in dwarf galaxies.
There are a handful of potential pathways which could give rise to this
empirical result.
\par
First, the mean star formation histories (SFHs) of galaxies vary with the
stellar mass of the system.
In semi-analytic models of galaxy formation (see, e.g., the reviews of
\citealt{Baugh2006} and~\citealt{Somerville2015a}), dwarf galaxies in the field
have more extended SFHs than their higher mass counterparts.
This mass dependence is also seen in hydrodynamical simulations of galaxy
formation~\citep[e.g.,][]{GarrisonKimmel2019}.
Since SN Ia delay-time distributions (DTDs) decline with age, galaxies with
more recent star formation should have higher specific SN Ia rates.
\par
Second,~\citet{Kistler2013} argued that the dependence of the specific SN Ia
rate on stellar mass may be driven by metallicity.
Lower mass galaxies host lower metallicity stellar populations
\citep{Gallazzi2005, Kirby2013} and lower metallicity gas reservoirs
(\citealp{Tremonti2004};~\citealp*{Zahid2011};~\citealp{Andrews2013,
Zahid2014}).
\citet{Kistler2013} point out that lower metallicity stars leave behind higher
mass WDs which could potentially grow to the Chandrasekhar mass and
subsequently explode more easily than their less massive counterparts.
Lower metallicity stars have weaker winds during the asymptotic giant branch
phase~\citep{Willson2000, Marigo2007}, leading to lower mass loss rates and
more massive cores~\citep{Kalirai2014}, producing more massive WDs for fixed
initial mass stars at lower metallcity (\citealp{Umeda1999};
\citealp*{Meng2008};~\citealp{Zhao2012}).
\par
Furthermore, in both the single~\citep[e.g.,][]{Whelan1973} and double
degenerate scenarios (e.g.,~\mbox{\citealp{Iben1984}};
\mbox{\citealp{Webbink1984}}), SNe Ia arise in binary systems.
Based on multiplicity measurements of Solar-type stars from the Apache Point
Observatory Galaxy Evolution Experiment~\citep[APOGEE;][]{Majewski2017},
\citet{Badenes2018} and~\citet*{Moe2019} find that the stellar close binary
fraction increases toward low metallicities.
% in APOGEE\footnote{
% 	Apache Point Observatory Galaxy Evolution Experiment
% }~\citep{Majewski2017},~\citet*{Moe2019} found that the stellar close binary
% fraction increases from~$\sim$10\% at~$\sim$$3Z_\odot$ where~$Z_\odot$ is the
% Solar metallicity to~$\sim$40\% at~$\sim$$0.1 Z_\odot$.
Consequently, dwarf galaxies should have more potential SN Ia progenitors per
unit mass of star formation due to more massive WDs and a higher close binary
fraction.
Motivated by these results,~\citet{Gandhi2022} explore a handful of
parametrizations for the metallicity dependence of the SN Ia rate in
re-simulated galaxies from FIRE-2~\citep{Hopkins2018}.
They find that a~$Z^{-0.5}$ scaling where~$Z$ is the metallicity by mass leads
to better agreement with the empirical relationship between galactic stellar
masses and stellar abundances than when using metallicity-independent SN Ia
rates.
\par
In this paper, we assume that the strong scaling of the specific SN Ia rate
with stellar mass is due to metallicity and conduct simple numerical
calculations to investigate its origin.
We combine the mean star formation histories of galaxies at fixed stellar
mass from the~\um~semi-analytic model~\citep{Behroozi2019} and the popular
$\tau^{-1}$ SN Ia DTD~\citep[e.g.,][]{Maoz2012a} with the mass-metallicity
relation (MZR) for galaxies~\citep{Tremonti2004, Andrews2013, Zahid2011,
Zahid2014}.
Given the mean SFH and DTD, we can compute the characteristic SN Ia rate for
galaxies of a given stellar mass, and by assuming that they lie along the
observed MZR, we can include various scalings of the rate with metallicity.
We describe the model in~\S~\ref{sec:galprops} and the effect on Type Ia rates
in~\S~\ref{sec:predictions}.
In~\S~\ref{sec:gce} we present simple models exploring the consequences of
metallicity-dependent SN Ia rates in galactic chemical evolution models.
We summarize our findings in~\S~\ref{sec:conclusions}.

\end{document}
