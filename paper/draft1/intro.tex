
\documentclass[ms.tex]{subfiles}
\begin{document}

\section{Introduction}
\label{sec:intro}

Supernova surveys have revealed that low-mass galaxies are more efficient
producers of type Ia supernovae (SNe Ia) than their higher mass counterparts
\citep[e.g.][]{Mannucci2005, Sullivan2006, Smith2012}.
These explosions arise from the thermonuclear detonation of a white dwarf
\cite[WD;][]{Hoyle1960}, the exposed carbon-oxygen core of a low-mass star.
Using archival spectroscopic data,~\citet{Brown2019} derived stellar masses for
the host galaxies of SNe Ia observed by the All Sky Automated Survey for
Supernovae~\citep[ASAS-SN;][]{Shappee2014, Kochanek2017}, finding that the
specific SN Ia rate -- the rate per unit stellar mass -- scales approximately
with the inverse square root of the mass itself (i.e.~$\dot{N}_\text{Ia} /
M_\star \sim M_\star^{-0.3}$).
\citet{Gandhi2022} demonstrate that this dependence is unfortunately plagued by
uncertainties in the galaxy stellar mass function (SMF; see, e.g.,
\citealp*{Weigel2016} for discussion of the systematics and challenges in
empirical measurements).
While~\citet{Brown2019} used the~\citet{Bell2003} SMF, the scaling becomes
shallower (approximately~$\dot{N}_\text{Ia} / M_\star \sim M_\star^{-0.3}$)
when using the steeper~\citet{Baldry2012} form due to a more abundant
underlying galaxy population at low stellar masses.
This leads to significantly better agreement between the ASAS-SN measurements
and Wiseman et al.'s~\citeyearpar{Wiseman2021} rates measured with SNe Ia from
the Dark Energy Survey~\citep[DES;][]{DES2016}, also normalized with the
\citet{Baldry2012} SMF.
Although there is debate surrounding the strength of the scaling, it is now
well-established qualitatively that the specific SN Ia rate is higher in dwarf
galaxies.
There are a handful of potential pathways which could give rise to this
empirical result.
\par
First, the mean star formation histories (SFHs) of galaxies are believed to
vary as a function of stellar mass. Semi-analytic models of galaxy formation
(see, e.g., the reviews of~\citealp{Baugh2006} and~\citealp{Somerville2015a})
suggest that dwarf galaxies in the field have more extended SFHs than their
higher mass counterparts.
Unless the SN Ia delay-time distribution~\citep[DTD; e.g.,][]{Maoz2012a}
depends on a galaxy's stellar mass in a highly specific manner, the variations
in SFHs should impact the intrinsic SN Ia rate across the various galaxy
demographics.
\par
Second,~\citet{Kistler2013} argue that an inverse dependence of the
specific SN Ia rate on stellar mass can be explained by considering metallicity
as a confounding variable.
Lower mass galaxies are empirically known to host lower metallicity stellar
populations~\citep{Gallazzi2005, Kirby2013} as a consequence of lower
metallicity gas reservoirs (\citealp{Tremonti2004};~\citealp*{Zahid2011};
\citealp{Andrews2013, Zahid2014}).
Whether or not a WD explodes as a SN Ia should be sufficiently decoupled from
its galactic environment, making the chemical composition of the progenitor
populations a more attractive explanation.
\citet{Kistler2013} postulate that this could arise because at fixed initial
mass, lower metallicity stars leave behind higher mass WDs, potentially making
it easier for them to grow to the Chandrasekhar mass and explode.
This arises due to the low Rosseland mean opacity of a low metallicity plasma,
which gives rise to weaker winds from asymptotic giant branch stars
\citep{Willson2000, Marigo2007}, leading to lower mass loss rates and
consequently stronger core growth~\citep{Kalirai2014}, thereby leaving behind a
more massive WD at fixed initial mass (\citealp{Umeda1999};~\citealp*{Meng2008};
\citealp{Zhao2012}).
\par
Furthermore, whether from the single-~\citep[e.g.][]{Whelan1973} or
double-degenerate scenario~\citep[e.g.][]{Iben1984, Webbink1984}, SNe Ia arise
in binary star systems.
Based on multiplicity measurements of solar-type stars in APOGEE\footnote{
	Apache Point Observatory Galaxy Evolution Experiment
}~\citep{Majewski2017}, the close binary fraction is known to increase with
decreasing metallicity (\citealp{Badenes2018};~\citealp*{Moe2019}).
Motivated by these results,~\citet{Gandhi2022} explore a handful of
parametrizations for the metallicity-dependence of the SN Ia rate in
re-simulated galaxies from FIRE-2~\citep{Hopkins2018}.
They find that a~$Z^{-0.5}$ scaling leads to better agreement with the observed
stellar mass - stellar metallicity relation than with metallicity-independent
rates.
They disfavor models with a stronger metallicity-dependence (e.g.~$\sim Z^{-1}$)
because the predictions of such models are in tension with the relationship
between the [Mg/Fe] and [Fe/H] abundances in M31 satellite galaxies observed
by~\citet*{Vargas2014}.
\par
At first glance, an inverse dependence of SN Ia rates on metallicity may seem
at odds with the recent realization by~\citet{Holoien2022} that dwarf galaxy
hosts of SNe Ia detected by ASAS-SN tend to be oxygen-rich relative to similar
mass galaxies.
However, SN hosts detected by any survey are likely not a representative sample
of the underlying galaxy population because, due to the steeply declining
nature of the DTD~\citep[e.g.][]{Maoz2012a}, the intrinsically highest SN Ia
rates should be in systems which experienced a recent starburst ($\lesssim$1
Gyr ago).
Since oxygen is produced mostly by massive stars with lifetimes shorter than
the delay-times of SNe Ia~\citep*[e.g.][]{Hurley2000, Johnson2019}, these
galaxies should also have a higher-than-average oxygen abundance (for
theoretical discussion of starbursts in galactic chemical evolution models,
see, e.g.,~\citealt{Johnson2020}).

\end{document}
