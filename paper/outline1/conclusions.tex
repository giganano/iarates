
\documentclass[ms.tex]{subfiles}
\begin{document}

\section{Discussion and Conclusions}
\label{sec:conclusions}

\begin{itemize}

	\item We have combined the mean SFHs of galaxies as predicted by
	the~\um~SAM~\citep{Behroozi2019} with the empirical MZR parametrized by
	\citet{Zahid2014} to investigate the origin of the dependence of the
	specific SN Ia rate on galaxy stellar mass.
	Empirical measurements of this trend are subject to uncertainties in the
	galaxy SMF at low stellar masses~\citep{Gandhi2022}.
	Within our framework, however, we are able to theoretically predict the
	scaling of the specific SN Ia rate with galaxy stellar mass in a manner
	that is independent of the SMF (see equation~\ref{eq:specia}).
	We find that the mean SFHs of galaxies of different stellar masses can
	account for only a factor of~$\sim$2 increase in the specific SN Ia rate
	between~$10^{10}$ and~$10^{7.2}~\msun$.
	If the rate scales with stellar mass as~$\dot{N}_\text{Ia} / M_\star
	\sim M_\star^{-0.3}$ as measured by~\citet{Gandhi2022} with the
	\citet{Baldry2012} SMF, then an additional metallicity-dependence of
	$Z^\gamma$ with~$\gamma \approx -0.5$ is required.
	Steeper scalings such as~$\dot{N}_\text{Ia} / M_\star \sim M_\star^{-0.5}$
	as derived by~\citet{Brown2019} with the~\citet{Bell2003} SMF require yet
	stronger scalings with metallicity (i.e.~$\gamma = 1 - 2$).
	Such a strong dependence on metallicity, however, is disfavored by
	\citet{Gandhi2022} as it fails to predict well-known correlations between
	elemental abundances in the~\textsc{FIRE-2} suite of cosmological zoom-in
	simulations~\citep{Hopkins2018}.
	The realization that SN Ia rates could depend on metallicity with a
	$\gamma = -0.5$ dependence should have important implications for chemical
	evolution models of dwarf galaxies.
	SNe Ia are responsible for a substantial portion of the abundances of
	iron-peak elements in the universe~\citep[e.g.][]{Johnson2019}, and
	incorporating more enrichment events at low metallicities could
	significantly impact the inferred evolutionary parameters in statistical
	fits to observed data.

	\item A scaling of~$\gamma = -0.5$ is in excellent agreement with the
	dependence of the close binary fraction measured in APOGEE~\citep{Moe2019}.
	This suggests that if a scaling of~$\dot{N}_\text{Ia} / M_\star \sim
	M_\star^{-0.3}$ is accurate, then the elevated SN Ia rates in dwarf
	galaxies can be explained by an increased binary fraction compared to their
	higher mass counterparts due to differences in their metal content.
	While~\citet{Gandhi2022} motivate their investigation from this viewpoint,
	here we take this argument one step further and postulate that this
	accounts for the~\textit{entire} increase in the specific SN Ia rate.
	This interpretation goes against the argument from~\citet{Kistler2013}
	which bases the increase in the SN Ia rate on lower metallicity stellar
	populations producing higher mass WDs~\citep{Umeda1999, Willson2000,
	Marigo2007, Meng2008, Zhao2012, Kalirai2014}.
	While our investigation cannot say anything definitive regarding the
	ongoing debate over the single- or double-degenerate progenitor scenario
	(see, e.g., the review in~\citealp{Maoz2014}), this interpretation somewhat
	disfavors the single-degenerate scenario.
	A natural explanation for higher mass WDs exploding more readily as SNe Ia
	appeals to continuous mass growth from a binary companion (i.e. the
	single-degenerate scenario) in which less accretion is required to reach
	the Chandrasekhar mass.
	However, a higher mass WD does not appear to be necessary to explain an
	$\dot{N}_\text{Ia} / M_\star \sim M_\star^{-0.3}$ scaling, instead
	suggesting that the majority of explosions are triggered by a mechanism
	in which the mass is irrelevant, such as a merger with another WD (i.e.
	the double-degenerate scenario).

	\item The calculations we have presented here are simplified in several
	regards.
	First, we assume the characteristic SFH predicted by a SAM at all stellar
	masses.
	This is reasonble for computing expected mean trends, but in principle, the
	underlying galaxy population should occupy a distribution of rates at
	fixed stellar mass due to variations in their SFHs.
	\citet{Brown2019} do not find strong evidence that the specific SN Ia rate
	depends on star formation activity, but we do not investigate this here
	because tracking galaxy-by-galaxy SFHs would considerably increase our
	computational expense.
	Second, we have assumed that all galaxies lie perfectly on the MZR.
	This should also be sufficient for computing mean trends, but similar to
	their SFHs, galaxies should populate a distribution of metallicities at
	fixed stellar mass.
	Third, our metallicity scaling of~$Z^{\gamma}$ where
	$Z = Z_\odot 10^\text{[O/H]}$ assumes that all SNe Ia in any galaxy arise
	from stellar populations at exactly the gas-phase abundance implied by the
	MZR.
	This is nonetheless a reasonable assumption because models of galactic
	chemical evolution incorporating an exchange of baryons with the
	surrounding medium via inflows and outflows suggest that galaxies should
	rapidly approach an equilibrium abundance in which the production of
	new metals is balanced by losses to star formation and outflows
	\citep{Larson1972, Weinberg2017}.
	This would imply that the young stellar populations which dominate the SN
	Ia rate (see Fig.~\ref{fig:sfhs} and discussion in~\S~\ref{sec:galprops})
	should be of similar metallicities -- at least on average.
	Although the close binary fraction is known to increase at low abundances
	in APOGEE~\citep{Badenes2018, Moe2019},~\citet{Mazzola2020} find that the
	binary fraction decreases for stars with high [$\alpha$/Fe] ratios at
	fixed [Fe/H], suggesting that the detailed chemical composition may
	significantly impact stellar multiplicity.
	Furthermore, the abundances of metals in stars is a many-dimensional
	quantity~\citep[e.g.][]{Ting2022}.
	However, a theoretical understanding of the metallicity dependence of SN Ia
	rates is a relatively new topic in the literature.
	A first-order understanding of how the rates scale with the absolute metal
	abundance in statistically average galaxies must be attained before
	considering how it should scale with the detailed chemical composition of
	the progenitor populations.

\end{itemize}

\end{document}
