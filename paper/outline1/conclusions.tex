
\documentclass[ms.tex]{subfiles}
\begin{document}

\section{Discussion and Conclusions}
\label{sec:conclusions}

\begin{itemize}

	\item We have combined the mean SFHs of galaxies as predicted by
	the~\um~SAM~\citep{Behroozi2019} with the empirical MZR parametrized by
	\citet{Zahid2014} to investigate the origin of the dependence of the
	specific SN Ia rate on galaxy stellar mass.
	Empirical measurements of this trend are subject to uncertainties in the
	galaxy SMF at low stellar masses~\citep{Gandhi2022}.
	Within our framework, however, we are able to theoretically predict the
	scaling of the specific SN Ia rate with galaxy stellar mass in a manner
	that is independent of the SMF (see equation~\ref{eq:specia}).
	We find that the mean SFHs of galaxies of different stellar masses can
	account for only a factor of~$\sim$2 increase in the specific SN Ia rate
	between~$10^{10}$ and~$10^{7.2}~\msun$.
	If the rate scales with stellar mass as~$\dot{N}_\text{Ia} / M_\star
	\sim M_\star^{-0.3}$ as measured by~\citet{Gandhi2022} with the
	\citet{Baldry2012} SMF, then an additional metallicity-dependence of
	$Z^\gamma$ with~$\gamma \approx -0.5$ is required.
	Steeper scalings such as~$\dot{N}_\text{Ia} / M_\star \sim M_\star^{-0.5}$
	as derived by~\citet{Brown2019} with the~\citet{Bell2003} SMF require yet
	stronger scalings with metallicity (i.e.~$\gamma = 1 - 2$).
	Such a strong dependence on metallicity, however, is disfavored by
	\citet{Gandhi2022} as it fails to predict well-known correlations between
	elemental abundances in the~\textsc{FIRE-2} suite of cosmological zoom-in
	simulations~\citep{Hopkins2018}.

	\item A scaling of~$\gamma = -0.5$ is in excellent agreement with the
	dependence of the close binary fraction measured in APOGEE~\citep{Moe2019}.
	This suggests that if a scaling of~$\dot{N}_\text{Ia} / M_\star \sim
	M_\star^{-0.3}$ is accurate, then the elevated SN Ia rates in dwarf
	galaxies can be explained by an increased binary fraction compared to their
	higher mass counterparts due to differences in their metal content.
	While~\citet{Gandhi2022} motivate their investigation from this viewpoint,
	here we take this argument one step further and postulate that this
	accounts for the~\textit{entire} increase in the specific SN Ia rate.
	This interpretation goes against the argument from~\citet{Kistler2013}
	which bases the increase in the SN Ia rate on lower metallicity stellar
	populations producing higher mass WDs~\citep{Umeda1999, Willson2000,
	Marigo2007, Meng2008, Zhao2012, Kalirai2014}.
	While our investigation cannot say anything definitive regarding the
	ongoing debate over the single- or double-degenerate progenitor scenario
	{\color{red} (refs)}, this interpretation somewhat disfavors the
	single-degenerate scenario.
	A natural explanation for higher mass WDs exploding more readily as SNe Ia
	appeals to continuous mass growth from a binary companion (i.e. the
	single-degenerate scenario) in which less accretion is required to reach
	the Chandrasekhar mass.
	However, a higher mass WD does not appear to be necessary to explain an
	$\dot{N}_\text{Ia} / M_\star \sim M_\star^{-0.3}$ scaling, instead
	suggesting that the majority of explosions are triggered by a mechanism
	in which the mass is irrelevant, such as a merger with another WD (i.e.
	the double-degenerate scenario).

	\item At first glance, an inverse dependence of SN Ia rates on metallicity
	may seem at odds with the recent realization by {\color{red} Holoien et al.
	(2020)} that ASAS-SN SN Ia host galaxies tend to be oxygen-rich relative to
	similar mass galaxies.
	This bias arises naturally in any SN survey, however, because the SN Ia
	rates are intrinsically higher in galaxies which experienced a recent
	starburst (of order~$\sim$1 Gyr ago), and post-starburst galaxies are
	known to be oxygen-rich due to enrichment by massive stars with short
	lifetimes {\color{red} (refs)}.

	\item The realization that SN Ia rates could depend on metallicity with a
	$\gamma = -0.5$ dependence has important implications for galactic chemical
	evolution.
	SN Ia are responsible for a substantial portion of the abundances of
	iron-peak elements in the universe {\color{red} (Johnson 2019)}, and
	incorporating more events at low metallicity could significantly impact the
	inferred evolutionary parameters.

\end{itemize}

\end{document}
