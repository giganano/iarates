
\documentclass[ms.tex]{subfiles}
\begin{document}

\section{Introduction}
\label{sec:intro}

\begin{itemize}

	\item Supernova surveys have revealed that low mass galaxies are
	more efficient producers of type Ia supernovae (SNe Ia) than their higher
	mass counterparts~\citep{Mannucci2005}.
	Invoking the correlation between a galaxy's stellar mass and its
	metallicity~\citep[e.g.][]{Andrews2013},~\citet{Kistler2013} proposed that
	this could arise if lower metallicity stellar populations give rise to
	more SN events at fixed mass.
	Using the All Sky Automated Survey for Supernovae (ASAS-SN),
	\citet{Brown2019} used archival spectroscopic data to derive stellar masses
	for SN Ia host galaxies and found that the specific SN Ia rate - the rate
	per unit stellar mass - scales with approximately with the inverse square
	root of the stellar mass (i.e.~$\dot{N}_\text{Ia} / M_\star \sim
	M_\star^{-0.5}$).
	
	\item This mass dependence, however, depends on how abundant the
	underlying galaxy population is at a given mass.
	\citet{Gandhi2022} demonstrated that this scaling becomes shallower
	(approximately~$\dot{N}_\text{Ia} / M_\star \sim M_\star^{-0.3}$) with
	the steeper~\citet{Baldry2012} stellar mass function than the
	\citet{Bell2003} form used by~\citet{Brown2019}.
	They demonstrate that this shallower dependence can be readily explained
	by a metallicity dependence that scales as~$Z^{-0.5}$ or~$Z^{-1}$, and
	that this does not significantly impact galaxy stellar masses and
	morphologies in the FIRE-2 cosmological zoom-in
	simulations~\citep{Hopkins2018}.
	
	\item \citet{Kistler2013} postulated that this metallicity-dependence can
	arise because lower metallicity stars produce higher mass white dwarfs at
	fixed ZAMS mass (refs), but this isn't the only possible explanation.
	Based on stellar multiplicity measurements from APOGEE\footnote{
		Apache Point Observatory Galaxy Evolution Experiment
		\citep{Majewski2017}.
	}, the stellar close binary fraction is known to increase with decreasing
	metallicity (\citealp{Badenes2018};~\citealp*{Moe2019}).
	Since SNe Ia are believed to arise from binary systems, this should mean
	that there are more progenitors at low metallicity and consequently in low
	mass galaxies.
	Low mass field galaxies also have more extended star formation histories
	than high mass galaxies (refs), and we demonstrate in~\S~X that this
	should be accompanied by a higher SN Ia rate at low redshift.
	
	\item In the present paper, we conduct order of magnitude calculations to
	investigate the sources of this scaling.


\end{itemize}

\end{document}

