
\documentclass[ms.tex]{subfiles}
\begin{document}

\section{Introduction}
\label{sec:intro}

\begin{itemize}

	\item Supernova surveys have revealed that low mass galaxies are
	more efficient producers of type Ia supernovae (SNe Ia) than their higher
	mass counterparts~\citep[e.g.][]{Mannucci2005}.
	Using archival spectroscopi data,~\citet{Brown2019} derived stellar
	masses for the host galaxies of SNe Ia observed by the All Sky
	Automated Survey for Supernovae (ASAS-SN), finding that the specific SN
	Ia rate (i.e. the rate per unit stellar mass) scales approximately with
	the inverse square root of the mass itself (i.e.
	$\dot{N}_\text{Ia} / M_\star \sim M_\star^{-0.5}$).
	This dependence, however, depends on how abundant the underlying galaxy
	population is at a given stellar mass.
	While~\citet{Brown2019} used the~\citet{Bell2003} stellar mass function
	(SMF),~\citet{Gandhi2022} demonstrate that the scaling becomes shallower
	(approximately~$\dot{N}_\text{Ia} / M_\star \sim M_\star^{-0.3}$) when
	using Baldry et al.'s~\citeyear{Baldry2012} form.
	This leads to significantly better agreement with Wiseman et al.'s
	\citeyear{Wiseman2021} rates measured with SNe Ia from the Dark Energy
	Survey (DES), also normalized with the~\citet{Baldry2012} SMF.
	The detailed origin of this effect is a topic of recent debate in
	the literature, and there are a handful of pathways which could give rise
	to a correlation between the observed SN rates and host galaxy properties.

	\item First, the mean star formation histories (SFHs) of galaxies are known
	to vary as a function of stellar mass.
	Semi-analytic models of galaxy formation (see, e.g., the reviews of
	\citealp{Baugh2006} and~\citealp*{Somerville2015a}) suggest that dwarf
	galaxies in the field have more extended SFHs than their higher mass
	counterparts.
	This, in principle, should impact the intrinsic SN rate of all spectral
	types across the various galaxy demographics.

	\item Second, due to the empirical correlation between a galaxy's stellar
	mass and its metallicity~\citep{Tremonti2004}, it is possible that the
	abundance of metals is a confounding variable between stellar mass and
	SN rates.
	\citet{Kobayashi1998} argued based on Galactic and cosmic chemical
	evolution models that SNe Ia should not occur below~$\sim$1/10th of the
	solar metallicity~$Z_\odot$.
	This argument is, however, challenged by the observation of SN Ia events in
	dwarf galaxies and in the low-metallicity outskirts of spirals
	\citep[e.g.][]{Prieto2008}.
	Stellar evolution models instead suggest that the SN Ia rate should
	increase at low metallicities.
	Due to the lower Rosseland mean opacity, winds from asymptotic giant branch
	(AGB) stars are weaker at lower metallicity, leading to lower mass loss
	rates and stronger core growth, eventually producing a higher mass WD at
	fixed initial mass (\citealp{Umeda1999, Willson2000, Marigo2007};
	\citealp*{Meng2008};~\citealp{Zhao2012, Kalirai2014}).
	\citet{Kistler2013} argue that this should make it easier for the white
	dwarf to reach the Chandrasekhar mass and consequently explode.

	\item This is, however, not the only pathway which could cause SN Ia rates
	to depend on the metallicity of the progenitor population.
	Regardless of whether they are produced by the single-
	\citep[e.g.][]{Whelan1973} or double-degenerate channel
	\citep[e.g.][]{Iben1984, Webbink1984}, SNe Ia arise from binary star
	systems.
	Based on multiplicity measurements of solar-type stars in APOGEE\footnote{
		Apache Point Observatory Galaxy Evolution Experiment
	}~\citep{Majewski2017}, the close binary fraction is known to increase
	at low metallicity (\citealp{Badenes2018};~\citealp*{Moe2019}).
	This suggests that there should be more SN Ia progenitors in low
	metallicity environments like dwarf galaxies.

	\item In the present paper, we conduct simple numerical calculations to
	investigate the origin of the specific SN Ia rate's dependence on stellar
	mass.
	We combine the relationship between galaxy stellar mass and mean SFH
	predicted by the~\textsc{UniverseMachine} semi-analytic model for galaxy
	formation~\citep{Behroozi2019} with the popular~$\tau^{-1}$ delay-time
	distribution (DTD) for SNe Ia~\citep[e.g.][]{Maoz2012} and empirical
	parametrizations of the mass-metallicity relation for galaxies
	\citep{Andrews2013, Zahid2014}.
	With a given SFH and DTD, we can compute the SN Ia rate for a galaxy of
	some stellar mass, and by assuming that it lies on the observed
	mass-metallicity relation, we can simultaneously fold in various
	assumptions about how the overall rate might scale with metallicity~$Z$.

\end{itemize}

\end{document}

